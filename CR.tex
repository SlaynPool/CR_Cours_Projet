
\documentclass{resume} % Use the custom resume.cls style
\usepackage[utf8]{inputenc}
\usepackage[T1]{fontenc}
\usepackage[left=0.75in,top=0.6in,right=0.75in,bottom=0.6in]{geometry} % Document margins
\newcommand{\tab}[1]{\hspace{.2667\textwidth}\rlap{#1}}
\newcommand{\itab}[1]{\hspace{0em}\rlap{#1}}
\name{Compte Rendu } % Your name
\address{Gestion de projet}
\begin{document}


\begin{rSection}{Mots Clé}

{\bf Délimitation (temps)} \newline  
{\bf Objéctif du projet} \newline
{\bf Etude des contraintes} \newline
{\bf Collaboration} \newline
{\bf Taches} \newline
{\bf CCTP} \newline
{\bf Echéances} \newline
{\bf MOE/MOA}
 
\end{rSection}

\begin{rSection}{Jalon}
 Pour les ASUR et uniquement pour les ASUR \newline 
 En Janvier  (GO NO GO) \newline
 Avril (Point sur la situation) \newline
 ....
 \end{rSection}
 
 
\begin{rSection}{Déroulement}
{\bf Réunion de lancement  -> Jalon 1 -> Jalon 2 -> Jalon n -> Bilan}

\end{rSection}

\begin{rSection}{Points abordés}
{\bf Partie prenante}\newline
{\bf Attention à l'effet tunel} \newline
Ne pas s'enfermer dans le projet

    Visibilité sur le projet | Respect du plan  \newline
     Travail en Equipe | Respect de l'equipe du projet \newline







Un projet ne doit jamais etre inutile. Si une personne soutien le projet il se passera bien mieux\\newlinew
S'assusrer que le projet soit bien compris par tous les acteurs\\newline



\end{rSection}

\begin{rSection}{Les Differents Outils Pour un Projet}
\begin{itemize}
    \item Diagrame de gant
    \item GIRA
    \item MS Projet
    \item Réunion
    \item Documentation
    \item Trace écrite
    \item Planning 
\end{itemize}

\end{rSection}
https://github.com/SlaynPool/CR\_Cours\_Projet\newline
C'est mon premier document en Latex !! 

\end{document}
